\documentclass[11pt,a4paper,oneside]{report}
%\usepackage[margin=0.25in]{geometry}
\usepackage{indentfirst}



\pagestyle{empty}

\renewcommand{\thesection}{1}

\begin{document}
\begin{center}
\LARGE{ \bf IROS 2012 Travel Award Reimbursement Packet: Experience Report }
%\Large{Daniel M. Lofaro \\ }
%\large{Electrical and Computer Engineering Dept. \\ Drexel University \\ Philadelphia PA, USA \\}
%\large{\tt dml46@drexel.edu }
\end{center}

\section*{Student Information}
\begin{itemize}
\item Name: Daniel M. Lofaro - Electrical and Computer Engineering Dept. Drexel University
\item Student Status: Full-time Ph.D Student (see attached file)
\item Status: United States Citizen (see attached file)
\item Email: \tt{dml46@drexel.edu}
\end{itemize}


\section*{Paper Information}
\begin{itemize}
\item Number: 1255
\item Title: \textit{Humanoid Throwing: Design of Collision-Free Trajectories with Sparse Reachable Maps}
\item Description: Design, simulation and hardware testing of a method to create collision-free throwing trajectories for high degree of freedom manipulators. 
\end{itemize}


\section*{Keynote Summary: Gill Pratt}
Being a part of the DARPA Robot Challenge (DRC) I was especially interested in Gill Pratt's Keynote talk.  
At the time our team (DRC-Hubo) had been tentatively accepted to the Track-A part of the competition with official acceptance to come on October 26th.  
The talk gave us a chance to hear Gill announce any further details or other competitors in Track-A.  
Though he did not state the names of any competitors at the time he did hint at foreign competitors that stepped down from their teaching positions to do this competition.  
This ended up being the SHAFT team from Tokyo University.  
In addition an important announcement was that the \textit{pump replacement} task with something else.  
This ended up being tentatively replaced by \textit{hose/cable insertion}.  
Other than the latter the rest of the presentation was about how DARPA is starting this competition as a direct response to the Fukushima disaster. 
He states that if one valve was turned the whole thing could have ended up different.
However we could not get humans close enough due to radiation levels.
He also mentioned that the robots we did send in could not traverse the stairs even through they were designed to do just that. 
This was because the robots were treaded robots and the stairs had front edges that were rounded off due to years of use.
Overall it was a nice informative talk

\section*{Summary of My Session}
I presented my work on \textit{Humanoid Throwing: Design of Collision-Free Trajectories with Sparse Reachable Maps}.  
This work shows how you can use a virtual model of a robot and pre-compute configurations that contain and do not contain self collisions.  
The configurations consist of end effector locations in free-space and the corresponding joint-space.  
The collection of N amount of configurations that do not contain a self collision form the Sparse Reachable Map or SRM.  
This map can be used in real-time to compute inverse kinematics on joints consisting of \textul{more than} 7-DOF.  
Though it can also be used for kinematic chains less than 7-DOF they can be analytically solved for thus is not as needed.
Over all I was happy with the presentation especially because one of my colleagues from Samsung was there.
He works on their \textit{``new''} humanoid called RoboRoy (which was officially daubed at IROS 2012).
He said that he sees the use for using an SRM for HRI or path planning of our high-DOF robots.

\end{document}