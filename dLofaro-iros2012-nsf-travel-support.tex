\documentclass[11pt,a4paper,oneside]{report}
\usepackage[margin=0.5in]{geometry}


\pagestyle{empty}

\renewcommand{\thesection}{1}

\begin{document}
\begin{center}
\LARGE{ \bf IROS 2012 Travel Fellowship Application }
%\Large{Daniel M. Lofaro \\ }
%\large{Electrical and Computer Engineering Dept. \\ Drexel University \\ Philadelphia PA, USA \\}
%\large{\tt dml46@drexel.edu }
\end{center}

\section*{Student Information}
\begin{itemize}
\item Name: Daniel M. Lofaro
\item Dept: Electrical and Computer Engineering
\item Univ: Drexel University
\item Student Status: Full-time Ph.D Student (see attached file)
\item Status: United States Citizen (see attached file)
\item Email: \tt{dml46@drexel.edu}
\end{itemize}


\section*{Paper Information}
\begin{itemize}
\item Number: 1255
\item Title: \textit{Humanoid Throwing: Design of Collision-Free Trajectories with Sparse Reachable Maps}
\item Description: Design, simulation and hardware testing of a method to create collision-free throwing trajectories for high degree of freedom manipulators. 
\end{itemize}


\section*{Expected Impact on Professional Development}
Daniel M. Lofaro expects to graduate in June 2012 and has an overarching goal of showcasing his work at IROS 2012 and find a Fulbright host.
Daniel was lead on the ICRA 2012 website creation, population and maintenance working closely with the general and program chair to ensure all information was properly distributed.
In addition Daniel worked with the RAS Student Activities Committee to create a \textit{student events} portion of ICRA 2012 with an ultimate goal of having the attendees be able to converse with their future colleagues in fun/interactive events and bring awareness of the RAS Student Association and its functions.  
The comedy act by robotisist turned funny-man Jorge Cham of Ph.D Comics was Daniel's brain child and crowning success of ICRA 2012.
In 2010 Daniel organized the workshop portion of the NATO-ASI three week conference/workshop in Cesme Turkey which included representatives from over 12 European countries.
The workshop was a resounding success and received compliments from attendees of all workshop sections.
The latter are Daniel's efforts to be an active and contributing member of the IEEE and robotics community.
Daniel is currently conversing with Laura Margheri (chair of the RAS Student Activities Committee for 2012-13) about the possibilities of a \textit{student events} section at IROS 2012.
Attending IROS will allow Daniel to continue his work with the RAS Student Activities Committee and help further his professional position in IEEE-RAS.

%From June 2011 to May 2012 Daniel worked as was working closely with the general and program chair of ICRA 2012 and was the lead
%is to become a teaching/research faculty at a university in the United States of America.  
%Daniel believes that the key for his and America's continuing success in science and engineering stems from successes with international research collaboration.
%Since 2008 Daniel has been focusing on improving his international collaboration, a desired trait in the Science, Technology, Engineering, and Mathematics (STEM) fields.
%The next critical step needed for Daniel to achieve his overarching goal is to become a Fulbright fellow giving him key international collaboration experience.  
%Becoming a Fulbright fellow will set Daniel apart from his peers of similar education and background.
%The expected impact of participating in IROS 2012 is allowing Daniel to find a foreign advisor willing to have him as a Fulbright fellow.
%IROS 2012 is an international event with numerous of Daniel's foreign colleagues in his field of study, thus it is the ideal setting for him to find an advisor for Fulbright.
%The new and strengthened relationships morphed from ICRA 2012 will play a key roll in Daniel's acceptance to the highly respected and competitive Fulbright fellowship program.
%The latter gives Daniel key experience needed for his professional development and achieving his overarching goal of becoming a teaching/research faculty at a university in the United States of America.


\section*{Project Funding}
\begin{itemize}
\item Agency: U.S. National Science Foundation (NSF)
\item Grant Title: \textit{Partnerships for International Research
and Education (PIRE)}
\item Grant Number: \#0730206
\end{itemize}

\section*{Travel Funding Sources}
Additional travel funding will be provided by the Drexel Autonomous Systems Lab (DASL).  If needed additional support will be provided by the Mechanical Engineering Department, the Electrical and Computer Engineering Department and the College of Engineering at Drexel University.


\end{document}